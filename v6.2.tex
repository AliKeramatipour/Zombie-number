%&pdflatex
\documentclass[1p]{elsarticle}

\usepackage{lineno,hyperref}
\usepackage{amsmath, amssymb, amscd, amsthm, amsfonts}
\usepackage{mathtools}
\modulolinenumbers[5]

\DeclarePairedDelimiter\ceil{\lceil}{\rceil} \DeclarePairedDelimiter\floor{\lfloor}{\rfloor}

\newtheorem{theorem}{Theorem}
\newtheorem{lemma}[theorem]{Lemma}
\newtheorem{conjecture}[theorem]{Conjecture}
\newtheorem{corollary}[theorem]{Corollary}
\newtheorem{example}[theorem]{Example}

\journal{Discrete Applied Mathematics}

\bibliographystyle{elsarticle-num}

\begin{document}
	
	\begin{frontmatter}
		
		\title{Zombie number}
		
		
		%% or include affiliations in footnotes:
		\author{Ali Keramatipour}
		\ead{alikeramatipour@ut.ac.ir}
		
		\author{Behnam Bahrak\corref{correspondingauthor}}
		\cortext[correspondingauthor]{Corresponding author}
		\ead{bahrak@ut.ac.ir}
		
		\address{School of Electrical and Computer Engineering, College of Engineering, University of Tehran, Tehran, Iran}
		
		\begin{abstract}
		{\it Zombies and Survivors} is a variant of the pursuit-evasion game {\it Cops and Robbers}, with the difference
		that zombies must always move closer to the survivor. The game is played on a simple graph by two players. The
		goal of the zombies is to catch the survivor while survivor's objective is to avoid being captured. The zombie
		number of $G$, denoted as $z(G)$, is the minimum number of zombies required to capture the survivor, no matter
		what moves survivor makes. In section \ref{conj-proof}, we prove a conjecture by Fitzpatrick et al.\cite{Fitz16} about the
		zombie number of the Cartesian product of two graphs.  This result provides a new proof for $z(Q_n) =
		\ceil*{\frac{2n}{3}}$. Next in section \ref{capturetime}, we introduce a new problem regarding {\it capture time}
		in Cartesian product of two graphs. At last in section \ref{np-capturetime} and \ref{np-zombienumber}, we prove
		two more open problems, we define {\it capture time} and {\it zombie number} problems more precisely and prove these
		problems belong to {\it NP-Hard} set of problems. 
		\end{abstract}
		
		\begin{keyword}
			Cartesian Product of Graphs\sep Zombie Number\sep Dominating Set\sep NP-Hard
		\end{keyword}
		
	\end{frontmatter}
	
\section{Introduction}\label{section-introduction}

The {\it Zombies and Survivors} game is played on a simple graph by two players. The deterministic version of this game
\cite{Fitz16} is played as follows. Initially, the zombie player chooses a number $z$ and places $z$ zombies on the
graph vertices. Then the survivor player chooses one single vertex which is the survivor's initial position. Starting
with the zombie player, on each turn, survivor player either moves to an adjacent vertex or stays at his current
location, while zombie player must move each zombie to one of its adjacent vertices so that they get closer to the
survivor. Here lies the difference between {\it Zombies and Survivor} and {\it Cops and Robbers} games, as in {\it Cops
and Robbers} cops should not necessarily get closer to the {\it Robber}, they can either hold their current position,
get closer, or further away from the robber. If any zombie and the survivor ever occupy the same vertex, the survivor is
captured and the zombie player wins. The zombie number of a graph $G$, denoted as $z(G)$, is the minimum number of
zombies required so that the zombie player always has a winning play on graph $G$. If a winning play exist, zombies
choose a set of initial vertices, and a move on their turn when they are faced with multiple, so that the survivor will
be captured no matter how he moves.

The Cartesian product $G \square H$ of two graphs $G$ and $H$, is a graph with vertex set of $V(G) \times V(H)$, where
vertices $(u_1 , u_2)$ and $(v_1 , v_2)$ are adjacent if and only if $u_1 = v_1$ and $ \{ u_2 , v_2 \} \in E_{H} $, or
$u_2 = v_2$ and $ \{u_1 , v_1 \} \in E_{G}$ \cite{West02}. Figure \ref{fig:p2} shows an example of the Cartesian product
of two graphs. We define $G_{i}$ as the induced graph by vertices $(u,v)$ in $G \square H$, where $v=i$. Similarly
$H_{j}$ is defined as the induced graph by vertices $(u,v)$ in $G \square H$, where $u=j$.

Capture time of a game, is the number of moves survivor can avoid being captured, if this can go to infinity, a survivor-win play exist.

NP-hardness (non-deterministic polynomial-time hardness) is, in computational complexity theory, the defining property
of a class of problems that are informally "at least as hard as the hardest problems in NP". A simple example of an
NP-hard problem is the dominating set problem in graph theory. A problem is assigned to the NP (nondeterministic polynomial time) class if
it is solvable in polynomial time by a nondeterministic Turing machine.

\begin{figure}[h!]
	\centering
	\includegraphics[width=0.9\linewidth]{fig/CpWest.png}
	\caption{$G \square H = C_3 \square C_4$ an example of a Cartesian Product}
	\label{fig:p2}
\end{figure}

In \cite{Fitz16}, Fitzpatrick et al. conjectured that $z(G \square H) \leq z(G) + z(H)$, and showed the correctness of
this inequality for specific cases of $G$ and $H$. In this paper we prove the conjecture for all graphs $G$ and $H$, and
use it to show that $z(Q_n) =  \ceil*{\frac{2n}{3}}$. 





\section{Conjecture and Proof}\label{conj-proof}

To prove $z(G \square H) \leq z(G) + z(H)$, we show that $z(G) + z(H)$ zombies are enough for the zombie player to
capture the survivor on $G \square H$.

To explain the proof we first need to define some notations. Assume $H$ and $G$ have $m$ and $n$ vertices, respectively.
In the Cartesian product of $G$ and $H$, each $G_{i}$  $(1 \leq i \leq m)$ is isomorphic to $G$, and each $H_{j}$ $(1
\leq j \leq n)$ is isomorphic to $H$. Figure \ref{fig:p1} illustrates this. We name the common vertex between $G_{i}$
and $H_{j}$, $(i,j)$. Also $(x,y)$ is the vertex where the survivor is located. A {\it G-move} is a move made on one of
the $G_{i}$'s edges.  Similarly, an {\it H-Move} is a move made on one of $H_{j}$'s edges. If survivor decides to saves
its current vertex, this move is considered both a {\it G-move} and an {\it H-move}. A {\it G-edge} is an edge in one of
the $G_{i}$s and an {\it H-edge} is an edge in one of the $H_{j}$s. A {\it winning state} is an state that zombie player
can catch survivor no matter what set of moves survivor makes, from that particular state. A $G$-equivalent graph, is
defined as a graph where each {\it G-zombie} in, say in vertex $(z_x,z_y)$, is placed on $z_x$ vertex and survivor is placed on
$x$ vertex. $H$-equivalent is also defined in the same way.


\begin{figure}[h!]
	\centering
	\includegraphics[width=0.9\textwidth]{fig/cp3.png}
	%\includegraphics[width=200pt]{cp.png}
	\caption{$G \square H$, $G_i$s, and $H_i$s .}
	\label{fig:p1}
\end{figure}









\begin{theorem}
	$z(G \square H) \leq z(G) + z(H)$.
\end{theorem}

\begin{proof}
	We provide a winning strategy for the Cartesian product of $G$ and $H$ using $z(G)+z(H)$ zombies. First, we place
	$z(G)$ zombies, that have a winning strategy on a single $G$, say $G_{1}$ and call them {\it G-zombies}. We do the
	same for $H_{1}$ and call them {\it H-zombies}. Initially all {\it G-zombies} share the same $G_{i}$. We name the
	subgraph $G_{i}$ shared by them $G_{z}$. We define $H_{s}$ in the same manner. As stated before, intially $z=s=1$.
	Consider one of the shortest paths (it does not matter which one we choose, we only need it to be unique) between
	vertices $z$ and $y$ in an $H$ and call it $p_H$. We also define $p_G$  in the same manner between $s$ and $x$. We
	call these paths' length, $d_H$ and $d_G$ respectively. Distance between each two vertices $(x,y)$ and $(u,v)$ in $G
	\square H$ is sum the of distances between their equivalents, $x$ and $u$ in $G$, and $y$ and $v$ in $H$.


	On each zombie turn, if $z \neq y$, each {\it G-zombie} will move along $d_H$ path in its corresponding $H$
	subgraph.  This move is valid since zombies' and survivor's equivalents on $H$ are getting closer. Now consider when $z =
	y$. Our {\it G-zombies} will only play their winning strategy (that they had on a single $G$) in this case. This
	move is also possible since in $G$'s strategy, zombies would get closer to survivor on each turn . If $z = y$ and
	survivor makes an {\it H-move}, {\it G-zombies} will only try to maintain their positioning by mimicing the exact
	same move on their corresponding $H$. This means for those turns that $z=y$ holds, if we consider the $G$-equivalent
	graph between {\it G-zombies} and survivor, it is just like a normal game played on a single $G$. Our {\it
	H-zombies} will follow the same strategy but in their corresponding environment.
	
	
	Suppose using this strategy $G \square H$ is {\it survivor-win}, then our survivor must do infinite moves in at
	least one direction. Without loss of generality, suppose the survivor makes infinite {\it H-moves}. We prove that
	this is not possible. After $d_G$ number of {\it H-moves}, our {\it H-zombies} will get to $H_x$. Now for each {\it
	G-move} made by survivor and having zombies chasing him, nothing changes in their $H$-equivalent graph. Since our
	survivor can do infinite {\it H-moves} and prevent being caught, it means that survivor could also avoid being
	caught on a single $H$ which contradicts our assumption.
	
\end{proof}
We now provide ab example for further understading:
\begin{example} $z(P_3 \square P_4 ) = 2$
	
\end{example}
\begin{figure}[h!]
	\centering
	\includegraphics[width=0.5\linewidth]{fig/p34m1.png}
	\caption{$P_3 \square P_4$ and initial vertices}
	\label{fig:p3}
\end{figure}
It's easy to show that $z(P_3) = z(P_4) = 1$. On each of these graphs, zombie's initial position could be any vertex of
our Path graph. For this example, we put our {\it G-zombie} and {\it H-zombie} ($G = P_3$ and $H = P_4$) both on vertex
$(1,1)$. We show our survivor with blue color, {\it H-zombie} with red, and {\it G-zombie} with green. {\it G-zombie}
will try to get to the same $G_{i}$ as survivor which is $G_3$ using an {\it H-edge} in an {\it HG-path} . {\it
H-zombie} will try to get to $H_3$. See Figure \ref{fig:p4}.
\begin{figure}[h!]
	\centering
	\includegraphics[width=0.5\linewidth]{fig/p34m2.png}
	\caption{First move of players}
	\label{fig:p4}
\end{figure}
After zombie's move our survivor must move. No matter what move he makes, either {\it G-zombie} has made itself closer
to $H_x$ or {\it H-zombie} has made itself closer to $G_y$. In this case, {\it H-zombie} got closer to  $H_x$. Since
neither {\it H or G-zombies}  share $H_x$ or $G_y$ with survivor, they will still try to achieve that.See Figure
\ref{fig:p5}.
\begin{figure}[h!]
	\centering
	\includegraphics[width=0.5\linewidth]{fig/p34m3.png}
	\caption{Second move made by zombies, third in total}
	\label{fig:p5}
\end{figure}
Now {\it H-zombie} shares the same copy of $H$ as survivor and its survivor's turn. If survivor moves to another $H_i$,
{\it H-zombie} will mimic the move. If survivor makes an {\it H-move} or saves his current vertex, {\it H-zombie} will
do whatever it did on a single $H$. This means survivor can not do infinite {\it H-moves}. Thus him being able to
survive he has to do infinite {\it G-moves}, which again leads to {\it G-zombie} capuring him. For other moves, you can
see Figure \ref{fig:p6}. We can not discuss each possible survivor move since they are a lot, but you can still easily
apply the strategy provided above. 
\begin{figure}[h!]
	\centering
	\includegraphics[width=0.6\linewidth]{fig/p34m6.png}
	\caption{Other moves made by players}
	\label{fig:p6}
\end{figure}

\begin{corollary}
	\label{C3}
	$z(Q_{n}) \leq \ceil*{\frac{2n}{3}}$
\end{corollary}
\begin{proof}
	We prove this by using both induction and the theorem proved above. First note that the Cartesian product of
	hypercube graphs $Q_{m}$ and $Q_{n}$ is equal to $Q_{m+n}$. It is easy to see $z(Q_3) = 2$, $z(Q_2) = 2$, and
	$z(Q_1) = 1$. For $n > 3$, we consider $Q_n$ as the Cartesian product of $Q_3$ and $Q_{n-3}$. Using the induction
	base, we know that $z(Q_{n-3}) \leq \ceil*{\frac{2n - 6}{3}}$.  According to the proved conjecture $z(Q_n) \leq
	z(Q_{n-3}) + z(Q_3)$ and $z(Q_{n-3}) \leq \ceil*{\frac{2n - 6}{3}} = \ceil*{\frac{2n}{3}} - 2$, we can see that
	$z(Q_n) \leq \ceil*{\frac{2n}{3}}$ .
\end{proof}

It is already proved that at least $\ceil*{\frac{2n}{3}}$ zombies are needed to capture one survivor on graph $Q_n$
(Theorem 16 of \cite{Fitz16}):

\begin{theorem}
	\label{T4}
	For each integer $n \geq 1$, $z(Q_n) \geq \ceil*{\frac{2n}{3}} $.
\end{theorem}

Combining {\it Corollary \ref{C3}} and {\it Theorem \ref{T4}} we can conclude that $z(Q_n)  = \ceil*{\frac{2n}{3}}$.
This proves Conjecture 18 from \cite{Fitz16} which is already proved in \cite{Offner19} with a different method. 
	

\section{Capture time and Cartesian product related result}\label{capturetime}
	We define two new parameters, $CT(G,z)$ (capture time) and $ZCT(G,k)$ (zombie capture time), where $G$ is a graph,
	and $z$ and $k$ are two integers.
	
	$CT(G,z)$ is the number of turns that survivor can avoid being caught, assuming that both players choose their best
	strategies. Zombie player will play with $z$ zombies and tries to make this number as least as possible, while
	survivor tries to maximize this.

	$ZCT(G,k)$ is the minimum number of {\it zombies} needed so that zombie player is able to capture the survivor in at
	most $k$ turns.

	Also $diam(G)$ is the length of $G$'s diamater.
	\begin{theorem}
		\label{T5}
		$CT( G \square H, Z_G + Z_H ) \leq diam(G) + diam(H) + CT(G, Z_G) + CT(H, Z_H)$
	\end{theorem}
	\begin{proof}
		We prove this by showing that survivor's {\it G-moves} cannot exceed $diam(H) + CT(G, Z_G)$. With the same
		conclusion, it can be shown that {\it H-moves} cannot exceed $diam(G) + CT(H, Z_H)$ as well.

		After first $diam(H)$ {\it G-moves} that survivor makes, $z$ = $y$ holds. Now since for each {\it G-move} made
		from now on by survivor, {\it G-zombies} can follow their strategy on a $G$ graph, after at most $CT(G,Z_G)$
		{\it G-moves}, survivor will be captured. 
		
		Since each of survivor's moves is either {\it G-move} or an {\it H-move} or even both (check definition above),
		total number of moves cannot exceed $diam(G) + diam(H) + CT(G, Z_G) + CT(H, Z_H)$.
	\end{proof}
\section{Zombie Capture time Problem is NP-Hard}\label{np-capturetime}
	We define zombie capture time ($ZCT_k$) problem as below:


	INSTANCE: Let $G = (V,E)$ be a simple undirected graph. Given graph $G$ and a positive integer $z$.


	QUESTION: Is $ZCT(G,k) \leq z$ ? In other words, can we capture the survivor after at most $k$ turns using $z$ zombies ?

	The {\it dominating set} problem is a well-known NP-Hard problem and is defined below:

	INSTANCE : Given a graph G and an integer z.

	QUESTION : Does G have a dominating set of size at most z ?

	\begin{theorem}
		$ZCT_k$ $\in$ NP-Hard
	\end{theorem}
	\begin{proof}
		We prove this by reducing a well-known NP-Hard problem (dominating set problem) to $ZCT_k$ in polynomial time.

		To have a better understading, consider the case when $k=1$. For zombie player being able to capture survivor in
		one move, every vertex not occupied by a zombie, should have a zombie neighbour. This is exactly the definition
		of a dominating set. Dominating set is a subset $D$ of $V(G)$ such that every vertex not in $D$ is adjacent to
		atleast one vertex of $D$. This simple proof shows that $ZCT_1 \in$ NP-Hard 

		Now consider $k$ to be an arbitrary number bigger than 1 and of $O(n^3)$. We construct a new graph $G'$ from
		$G$. Suppose $G$ has $n$ vertices. For each vertex $v \in V(G)$, we add a {\it new} path with $k$ vertices
		ending in $v$. \ref{fig:p7} We name each {\it new} vertex $(v,i)$ for $1 \leq i \leq k - 1$.
		
		\begin{figure}[h!]
			\centering
			\includegraphics[width=0.9\linewidth]{fig/ZCT.png}
			\caption{$G'$ obtained from G where for $k = 1,2,3$}
			\label{fig:p7}
		\end{figure}		


		$V(G')$ is of $O(nk)$ which is of $O(n^4)$. Thus creating $G'$ can be done in polynomial time. Now we solve
		$ZCT_k$ on graph $G'$ and get a number $z$, and a set $S$ of vertices as our zombies' intial positions. We build
		a set $DS$ on $G$ from set $S$ on $G'$. For each vertex $v \in S$ or $(v,i) \in S$ add $v$ to $DS$. Now suppose
		$DS$ is not a dominating set, thus there is a vertex $u$ not dominated by $DS$. This means there was no zombie
		on vertices $u$ and $(u,i)$ on graph $G'$. By having our survivor on $(u,k-1)$, there is no zombie with distance
		of $k$ or less from him which means that he will not be captured if he doesn't move. Thus, $DS$ must be a
		dominating set and $ |minimumDominatingSet(G)| \leq |DS| \leq |S| \leq z$.

		Now consider minimum dominating set of $G$, set $MDS$. For each vertex $v \in MDS$ place a zombie on vertex $v$
		of $G'$. These zombies can capture the survivor in at most $k$ moves. Consider survivor's intial vertex. It
		should not be on {\it old} vertices of $G'$ since they are all dominated by our zombies and survivor would be
		captured in one move. Suppose survivor is initially on $(u,i)$. If $u$ is occupied by a zombie already, that
		zombie will move along the $u-path$ and since our survivor is trapped, he will be captured within $k$ moves. If
		there is no zombie on $u$ there is a vertex $u'$ occupied zombie which $ \{u',u\} \in E(G)$. On its move, zombie
		will move to $u$. Thus our survivor is trapped again and will be captured withing $k$ moves. Therefore, $z \leq
		|minimumDominatingSet(G)|$.

		By combining results, $z = |minimumDominatingSet(G)|$. Therefore dominating set problem is reduced to $ZCT_k$.
		Note that we assumed $k$ is of $O(n^3)$. In next section \ref{np-zombienumber}, we prove this assumption changes nothing.

	\end{proof}

\section{Zombie Number Problem is NP-Hard}\label{np-zombienumber}

	\begin{lemma}
		If survivor can avoid being captured after $(n + 1) * n^2$ moves, he can avoid being captured forever.
	\end{lemma}
	\begin{proof}
		Define $zombieDist$ as sum of the distances between each zombie and survivor. It is not hard to see that after 2
		rounds of play, that is each player has played once, $zombieDist$ won't increase, since zombies are always
		getting closer. Now we show that if $zombieDist$ does not strictly decrease after each player takes turn for $n
		+ 1$ times, survivor can avoid being captured forever. Consider the sequence of vertices occupied by survivor in
		last $n + 1$ move. By pigenhole principle, one vertex has been seen by survivor atleast twice. If survivor keeps
		repeating those moves, he will maintain his distance from zombies and will avoid being captured forever.

		Therefore, for a graph $G$ which zombie has a strategy to win, after each $(n + 1)$ move, $zombieDist$ should
		strictly decrease. $zombieDist$ is at most $n^2$, since there is not more than $n$ zombies and each zombie is at
		distance at most $n$ from survivor (this bound can be easily improved to $n * diam(G)$ but we only need to be of
		$O(n)$). Thus, after $(n + 1) * n^2$ moves, zombies would capture the survivor.
	\end{proof}

	By using this lemma, we can see $ZCK_k$ problems for $k > (n + 1) * n^2$ are as the same as the problem for $ZCK_{(n
	+ 1) * n^2}$. Thus we have proved $ZCK_k \in$ NP-Hard for every positive $k$.

	Now we define zombie number ($Z$) problem:

	INSTANCE: Let $G = (V,E)$ be a simple undirected graph. Given graph $G$.

	QUESTION: Is $Z(G) \leq z$ ?

	\begin{theorem}
		$Z \in$ NP-Hard
	\end{theorem}
	\begin{proof}
		We reduce $ZCK_{(n + 1) * n ^ 2}$ problem to $Z$.

		The $(n + 1) * n ^ 2$ plays the role of $\infty$ in this problem. If there is a zombie-win play by a number of
		zombies for $Z$ problem, they would be able to capture him before having our 'turn counter' reach $(n + 1) * n
		^ 2$. This shows that $ZCK_{(n + 1) * n ^ 2}$ is reducible to
		$Z$. Therefore, $Z \in$ NP-Hard.
	\end{proof}

	By doing some minor changes, proof would apply to $Cops and Robber$ game as well.  {\it
	Cop-number} $c(G)$ 

\begin{thebibliography}{999}
	
	\bibitem{Fitz16}  
	Fitzpatrick, Shannon L., J. Howell, Margaret-Ellen Messinger, and David A. Pike. "A deterministic version of the
	game of zombies and survivors on graphs." Discrete Applied Mathematics 213 (2016): 1-12.
	\bibitem{Offner19}
	Offner, David, and Kerry Ojakian. "Comparing the power of cops to zombies in pursuit-evasion games." Discrete
	Applied Mathematics (2019).
	\bibitem{West02}
	West, Douglas B. "Introduction to Graph Theory." Prentice hall, (1996).
	\bibitem{reviewer}
	One of our reviewer's notes, Discrete Applied Mathematics, (2020).
\end{thebibliography}
	
\end{document}